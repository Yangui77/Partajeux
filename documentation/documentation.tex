\documentclass[a4paper,12pt,abstracton,titlepage]{scrartcl}

\usepackage[french]{babel}
%\usepackage[T1]{fontenc}
\usepackage[utf8]{inputenc} % Umlaute, evtl. vom Betriebssystem abhaengig
\usepackage{lmodern}
\usepackage{pgfgantt}
\usepackage{titlesec}
\usepackage{float}
\usepackage{floatflt}
\usepackage{blindtext}
\usepackage{amsmath}
\usepackage{tabularx,url}
\usepackage[a4paper, left=2cm, textwidth=17cm, top=1.5cm]{geometry}
\usepackage{hyperref}
\usepackage{longtable}


\titleformat*{\section}{\large\bfseries}
\titleformat*{\subsection}{\large\bfseries}
\titleformat*{\subsubsection}{\large\bfseries}
\titleformat*{\paragraph}{\large\bfseries}
\titleformat*{\subparagraph}{\large\bfseries}

\renewcaptionname{french}{\figurename}{Fig.}

%\titlehead{Ulm University}
%\title{Title}
%\subject{Subject}
%\author{Author}
%\publishers{%
	%\rule{\textwidth}{0.4pt} \\
	%\vspace{0.5cm}
    %\normalfont\normalsize%
    %\parbox{0.9\linewidth}{%
    %    Abstract or Introduction
    %} \\
    %\vspace{0.5cm}
   	%\rule{\textwidth}{0.4pt}
%}

\renewcommand*\contentsname{Summary}

\begin{document}
%\maketitle

%%% begin costom title
{\Large\noindent \emph{ESIEE Paris}}

{\Large\noindent \emph{SI-4301B}}
\vspace{1cm}
\begin{center}
	{\huge \textbf{Partajeux}
	\\
	\vspace{0.3cm}
	\large DANG Hung Hoang, FUDITPHU Yan-Guillaume, KULZER Ulrike
	\\
	\vspace{0.2cm}
	\today}
\end{center}
%%% end custom title
\vspace{1cm}
%\maketitle
\tableofcontents

\setcounter{page}{1} % reset page counter to one for the first page, leave the title page out

% ################### guide d'utilisateur #######################
\newpage
\section{Présentation du projet}
\subsection{Introduction/Description}
\subsection{Spécifications}
\section{Analyse fonctionnelle}
\section{Modèle Conceptuel de données}
% E-R-Diagramme plus BDD finale
\section{Documentation}
\subsection{Documentation technique}
\subsection{Documentation utilisateurs}
% captures d'écran

\\
\begin{minipage}[t]{0.5\textwidth}
    \begin{itemize}
    \item \textit{le chiffre de César :}\\

    \paragraph{}
    \item \textit{le chiffre de Vigenère :}\\
    

    \end{itemize}
 %   \begin{center}
  %  \raggedleft
% 	\includegraphics[height=3.4cm]{}
 	%\caption{exemple du cryptage de Vigenère}
% 	\label{exVig}
%    \end{center} 
	
  \end{minipage} 
%  \begin{minipage}[t]{0.4\linewidth}
 %   \raggedleft
  %  \strut\vspace*{-\baselineskip}\newline\includegraphics[width=0.9\linewidth]{}
    %\caption{décalage de l'alphabet}
   % \label{cesar}
    %\paragraph{}
    %\strut\vspace*{-\baselineskip}\newline\includegraphics[width=0.9\linewidth]{}
    %\caption{tableau de Vigenère}
    %\label{tabVig}
  %\end{minipage}

\begin{minipage}[t]{0.5\textwidth}
    \begin{itemize}
    \item \textit{celui de la machine Enigma :}\\
    
    \end{itemize}
  \end{minipage}

\newpage
\subsection{Fonctionnement du programme}

\subsubsection{Accueil et choix de la langue}
Pour exécuter notre programme, il est nécessaire de l'ouvrir avec le logiciel Pycharm, d'aller dans le fichier "main\_module" et d'appuyer sur "run" - "run" - "run main\_module".
Les instructions du programme seront affichées dans la console qui s'ouvre en bas de l'interface comme montrés ci dessous :
\vspace{0.5cm}
\\
%{\fbox{\parbox{\textwidth}{\raggedright
%\includegraphics{}}}
%	\captionof{figure}{Écran d'accueil}
%	\label{SP}
%}
\vspace{0.5cm}
Vous venez d’entrer dans le programme Cryptographie, vous devez entrer la lettre ' f ' pour avoir les instructions en français et ' e ' pour les avoir en anglais.
Si la lettre entrée est ' e ', les prochaines instructions seront donc en anglais. \\
\\
Note: Vous retrouverez les mêmes interfaces en français.\\
\\
Une nouvelle fenêtre va ensuite apparaître :
\vspace{0.5cm}
\\



\newpage
\subsection{Structure du programme}
% diagrammme
\vspace{1cm}
%\begin{minipage}[c]{\textwidth}
%\centering
%    \includegraphics[width=\textwidth, trim=1mm 50mm 1mm 1mm, clip]{}
    %TRIM = LINKS UNTEN RECHTS OBEN
%    \captionof{figure}{Structure du programme en modules}
%    \label{img:structure}
%\end{minipage}



\section{Annexe: Code du programme}
{\large
\textit{L'ordre des modules:}
\begin{enumerate}
\item main\_module
\item screen\_module
\item screen\_constants
\item text\_module
\item tests
\end{enumerate}
}





%\begin{floatingfigure}[l]{5.3cm}
%	\includegraphics[width=0.25\textwidth]{./Bilder/2venn.png}
%	\caption{Prinzip der additiven Farbmischung}
%	\label{venn}
%\end{floatingfigure}


%\par
%\vspace{2em}
%\vspace{1em}

% "`Standard Definition Television"', HDTV für "`High Definition Television'".


%\begin{itemize}
%\item \blindtext
%\item \blindtext
%\end{itemize}
%\begin{enumerate}
%\item \blindtext
%\item \blindtext
%\end{enumerate}
%\begin{description}
%\item [Ant] \blindtext
%\item [Elephant] \blindtext

%\textbf{greatest} 
%\underline{science} 
%\textbf{\textit{accident}}.


\end{document}




% LABEL UNTER CAPTION

